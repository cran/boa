\HeaderA{boa.hpd}{Highest Probability Density Interval}{boa.hpd}
\keyword{univar}{boa.hpd}
\begin{Description}\relax
Estimates the highest probability density (HPD) interval for the given
parameter draws. Uses the Chen and Shao algorithm assuming a unimodal
marginal posterior distribution.
\end{Description}
\begin{Usage}
\begin{verbatim}
boa.hpd(x, alpha)
\end{verbatim}
\end{Usage}
\begin{Arguments}
\begin{ldescription}
\item[\code{x}] MCMC draws from the marginal posterior to use in computing the HPD.
\item[\code{alpha}] Specifies the 100*(1 - alpha)\% interal to compute.
\end{ldescription}
\end{Arguments}
\begin{Value}
A vector containing the lower and upper bound of the HPD interval, labeled
"Lower Bound" and "Upper Bound", respectively.
\end{Value}
\begin{Author}\relax
Brian J. Smith
\end{Author}
\begin{References}\relax
Chen, M-H. and Shao, Q-M. (1999). Monte Carlo estimation of Bayesian
credible and HPD intervals. Journal of Computational and Graphical Statistics,
8(1), 69-92.
\end{References}

