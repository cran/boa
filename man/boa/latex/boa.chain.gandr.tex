\HeaderA{boa.chain.gandr}{Gelman and Rubin Convergence Diagnostics}{boa.chain.gandr}
\keyword{univar}{boa.chain.gandr}
\begin{Description}\relax
Computes the Gelman and Rubin convergence diagnostics for a list of MCMC
sequences. Estimates are calculated from the second half of each sequence.
\end{Description}
\begin{Usage}
\begin{verbatim}
boa.chain.gandr(chain, chain.support, alpha, pnames, window, to)
\end{verbatim}
\end{Usage}
\begin{Arguments}
\begin{ldescription}
\item[\code{chain}] List of matrices whose columns and rows contain the monitored
parameters and the MCMC iterations, respectively. The iteration numbers and
parameter names must be assigned to the dimnames.
\item[\code{chain.support}] List of matrices whose columns and rows contain the
monitored parameters and the support (lower and upper limits), respectively.
\item[\code{alpha}] Quantile (1 - alpha / 2) at which to estimate the upper limit of
the shrink factor.
\item[\code{pnames}] Character vector giving the names of the parameters to use in
the analysis. If omitted, all parameters are used.
\item[\code{window}] Proportion of interations to include in the analysis. If
omitted, 50\% are included.
\item[\code{to}] Largest iteration to include in the analysis. If omitted, no upper
bound is set.
\end{ldescription}
\end{Arguments}
\begin{Value}
\begin{ldescription}
\item[\code{psrf}] A vector containing the Gelman and Rubin (uncorrected) potential
scale reduction factors for the monitored parameters.
\item[\code{csrf}] A matrix whose columns and rows are the Gelman and Rubin corrected
scale reduction factors (i.e. shrink factor estimates at the median and
specified quantile of the sampling distribution) and the monitored parameters,
respectively. A correction of (df + 3) / (df + 1) is applied to the scale
reduction factors.
\item[\code{mpsrf}] A numeric value giving the multivariate potential scale reduction
factor proposed by Brooks and Gelman.
\item[\code{window}] A numeric vector with two elements giving the range of the
iterations used in the analysis.
\end{ldescription}
\end{Value}
\begin{Author}\relax
Brian J. Smith, Nicky Best, Kate Cowles
\end{Author}
\begin{References}\relax
\Enumerate{
\item Brooks, S. and Gelman, A. (1998). General methods for monitoring
convergence of iterative simulations. Journal of Computational and Graphical
Statistics, 7(4), 434-55.
\item Gelman, A. and Rubin, D. B. (1992). Inference from iterative
simulation using multiple sequences. Statistical Science, 7, 457-72.
}
\end{References}
\begin{SeeAlso}\relax
\code{\LinkA{boa.plot}{boa.plot}}, \code{\LinkA{boa.plot.bandg}{boa.plot.bandg}},
\code{\LinkA{boa.plot.gandr}{boa.plot.gandr}}, \code{\LinkA{boa.print.gandr}{boa.print.gandr}}
\end{SeeAlso}

