\HeaderA{boa.handw}{Heidelberger and Welch Convergence Diagnostics}{boa.handw}
\keyword{univar}{boa.handw}
\begin{Description}\relax
Computes the Heidleberger and Welch convergence diagnostics for the
parameters in an MCMC sequence.
\end{Description}
\begin{Usage}
\begin{verbatim}
boa.handw(link, error, alpha)
\end{verbatim}
\end{Usage}
\begin{Arguments}
\begin{ldescription}
\item[\code{link}] Matrix whose columns and rows contain the monitored parameters
and the MCMC iterations, respectively. The iteration numbers and parameter
names must be assigned to \code{dimnames(link)}.
\item[\code{error}] Accuracy of the posterior estimates for the parameters.
\item[\code{alpha}] Alpha level for the confidence in the sample mean of the
retained iterations.
\end{ldescription}
\end{Arguments}
\begin{Value}
A matrix whose columns and rows are the Heidleberger and Welch convergence
diagnostics (i.e. stationarity test, number of iterations to keep and to drop,
Cramer-von-Mises statistic, halfwidth test, mean, and halfwidth) and the
monitored parameters, respectively.
\end{Value}
\begin{Author}\relax
Brian J. Smith, Nicky Best, Kate Cowles
\end{Author}
\begin{References}\relax
Heidelberger, P. and Welch, P. (1983). Simulation run length control
in the presence of an initial transient. Operations Research, 31, 1109-44.
\end{References}
\begin{SeeAlso}\relax
\code{\LinkA{boa.print.handw}{boa.print.handw}}
\end{SeeAlso}

