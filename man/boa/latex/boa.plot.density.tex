\HeaderA{boa.plot.density}{Plot Density Functions}{boa.plot.density}
\keyword{hplot}{boa.plot.density}
\begin{Description}\relax
Estimates and displays, in a single plot, the density function(s) for the 
specified parameter(s).
\end{Description}
\begin{Usage}
\begin{verbatim}
boa.plot.density(lnames, pname, bandwidth = boa.par("bandwidth"),
window = boa.par("kernel"), annotate = boa.par("legend"))
\end{verbatim}
\end{Usage}
\begin{Arguments}
\begin{ldescription}
\item[\code{lnames}] Character vector giving the names of the desired MCMC sequence 
in the working session list of sequences.
\item[\code{pname}] Character string giving the name of the parameter to be plotted.
\item[\code{bandwidth}] Function for computing the bandwidth used in estimating the 
density functions for parameters. This should take one argument which is a 
numeric vector of data on which density estimation is to be performed. A 
constant bandwidth may be specified by having this function return the desired 
constant.
\item[\code{window}] Character string giving the type of window used in estimating 
the density functions for the parameters. Available choices are "cosine", 
"gaussian", "rectangular", or "triangular".
\item[\code{annotate}] Logical value indicating that a legend be included in the plot.
\end{ldescription}
\end{Arguments}
\begin{Value}
A logical value indicating that the plot was successfully created.
\end{Value}
\begin{Author}\relax
Brian J. Smith
\end{Author}
\begin{SeeAlso}\relax
\code{\LinkA{boa.plot}{boa.plot}}
\end{SeeAlso}

